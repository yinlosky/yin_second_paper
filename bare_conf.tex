\documentclass[conference]{IEEEtran}
\usepackage{graphicx}





% Some very useful LaTeX packages include:
% (uncomment the ones you want to load)


% *** MISC UTILITY PACKAGES ***
%
%\usepackage{ifpdf}
% Heiko Oberdiek's ifpdf.sty is very useful if you need conditional
% compilation based on whether the output is pdf or dvi.
% usage:
% \ifpdf
%   % pdf code
% \else
%   % dvi code
% \fi
% The latest version of ifpdf.sty can be obtained from:
% http://www.ctan.org/tex-archive/macros/latex/contrib/oberdiek/
% Also, note that IEEEtran.cls V1.7 and later provides a builtin
% \ifCLASSINFOpdf conditional that works the same way.
% When switching from latex to pdflatex and vice-versa, the compiler may
% have to be run twice to clear warning/error messages.






% *** CITATION PACKAGES ***
%
%\usepackage{cite}
% cite.sty was written by Donald Arseneau
% V1.6 and later of IEEEtran pre-defines the format of the cite.sty package
% \cite{} output to follow that of IEEE. Loading the cite package will
% result in citation numbers being automatically sorted and properly
% "compressed/ranged". e.g., [1], [9], [2], [7], [5], [6] without using
% cite.sty will become [1], [2], [5]--[7], [9] using cite.sty. cite.sty's
% \cite will automatically add leading space, if needed. Use cite.sty's
% noadjust option (cite.sty V3.8 and later) if you want to turn this off.
% cite.sty is already installed on most LaTeX systems. Be sure and use
% version 4.0 (2003-05-27) and later if using hyperref.sty. cite.sty does
% not currently provide for hyperlinked citations.
% The latest version can be obtained at:
% http://www.ctan.org/tex-archive/macros/latex/contrib/cite/
% The documentation is contained in the cite.sty file itself.






% *** GRAPHICS RELATED PACKAGES ***
%
\ifCLASSINFOpdf
  % \usepackage[pdftex]{graphicx}
  % declare the path(s) where your graphic files are
  % \graphicspath{{../pdf/}{../jpeg/}}
  % and their extensions so you won't have to specify these with
  % every instance of \includegraphics
  % \DeclareGraphicsExtensions{.pdf,.jpeg,.png}
\else
  % or other class option (dvipsone, dvipdf, if not using dvips). graphicx
  % will default to the driver specified in the system graphics.cfg if no
  % driver is specified.
  % \usepackage[dvips]{graphicx}
  % declare the path(s) where your graphic files are
  % \graphicspath{{../eps/}}
  % and their extensions so you won't have to specify these with
  % every instance of \includegraphics
  % \DeclareGraphicsExtensions{.eps}
\fi
% graphicx was written by David Carlisle and Sebastian Rahtz. It is
% required if you want graphics, photos, etc. graphicx.sty is already
% installed on most LaTeX systems. The latest version and documentation can
% be obtained at: 
% http://www.ctan.org/tex-archive/macros/latex/required/graphics/
% Another good source of documentation is "Using Imported Graphics in
% LaTeX2e" by Keith Reckdahl which can be found as epslatex.ps or
% epslatex.pdf at: http://www.ctan.org/tex-archive/info/
%
% latex, and pdflatex in dvi mode, support graphics in encapsulated
% postscript (.eps) format. pdflatex in pdf mode supports graphics
% in .pdf, .jpeg, .png and .mps (metapost) formats. Users should ensure
% that all non-photo figures use a vector format (.eps, .pdf, .mps) and
% not a bitmapped formats (.jpeg, .png). IEEE frowns on bitmapped formats
% which can result in "jaggedy"/blurry rendering of lines and letters as
% well as large increases in file sizes.
%
% You can find documentation about the pdfTeX application at:
% http://www.tug.org/applications/pdftex





% *** MATH PACKAGES ***
%
%\usepackage[cmex10]{amsmath}
% A popular package from the American Mathematical Society that provides
% many useful and powerful commands for dealing with mathematics. If using
% it, be sure to load this package with the cmex10 option to ensure that
% only type 1 fonts will utilized at all point sizes. Without this option,
% it is possible that some math symbols, particularly those within
% footnotes, will be rendered in bitmap form which will result in a
% document that can not be IEEE Xplore compliant!
%
% Also, note that the amsmath package sets \interdisplaylinepenalty to 10000
% thus preventing page breaks from occurring within multiline equations. Use:
%\interdisplaylinepenalty=2500
% after loading amsmath to restore such page breaks as IEEEtran.cls normally
% does. amsmath.sty is already installed on most LaTeX systems. The latest
% version and documentation can be obtained at:
% http://www.ctan.org/tex-archive/macros/latex/required/amslatex/math/





% *** SPECIALIZED LIST PACKAGES ***
%
%\usepackage{algorithmic}
% algorithmic.sty was written by Peter Williams and Rogerio Brito.
% This package provides an algorithmic environment fo describing algorithms.
% You can use the algorithmic environment in-text or within a figure
% environment to provide for a floating algorithm. Do NOT use the algorithm
% floating environment provided by algorithm.sty (by the same authors) or
% algorithm2e.sty (by Christophe Fiorio) as IEEE does not use dedicated
% algorithm float types and packages that provide these will not provide
% correct IEEE style captions. The latest version and documentation of
% algorithmic.sty can be obtained at:
% http://www.ctan.org/tex-archive/macros/latex/contrib/algorithms/
% There is also a support site at:
% http://algorithms.berlios.de/index.html
% Also of interest may be the (relatively newer and more customizable)
% algorithmicx.sty package by Szasz Janos:
% http://www.ctan.org/tex-archive/macros/latex/contrib/algorithmicx/




% *** ALIGNMENT PACKAGES ***
%
%\usepackage{array}
% Frank Mittelbach's and David Carlisle's array.sty patches and improves
% the standard LaTeX2e array and tabular environments to provide better
% appearance and additional user controls. As the default LaTeX2e table
% generation code is lacking to the point of almost being broken with
% respect to the quality of the end results, all users are strongly
% advised to use an enhanced (at the very least that provided by array.sty)
% set of table tools. array.sty is already installed on most systems. The
% latest version and documentation can be obtained at:
% http://www.ctan.org/tex-archive/macros/latex/required/tools/


%\usepackage{mdwmath}
%\usepackage{mdwtab}
% Also highly recommended is Mark Wooding's extremely powerful MDW tools,
% especially mdwmath.sty and mdwtab.sty which are used to format equations
% and tables, respectively. The MDWtools set is already installed on most
% LaTeX systems. The lastest version and documentation is available at:
% http://www.ctan.org/tex-archive/macros/latex/contrib/mdwtools/


% IEEEtran contains the IEEEeqnarray family of commands that can be used to
% generate multiline equations as well as matrices, tables, etc., of high
% quality.


%\usepackage{eqparbox}
% Also of notable interest is Scott Pakin's eqparbox package for creating
% (automatically sized) equal width boxes - aka "natural width parboxes".
% Available at:
% http://www.ctan.org/tex-archive/macros/latex/contrib/eqparbox/





% *** SUBFIGURE PACKAGES ***
%\usepackage[tight,footnotesize]{subfigure}
% subfigure.sty was written by Steven Douglas Cochran. This package makes it
% easy to put subfigures in your figures. e.g., "Figure 1a and 1b". For IEEE
% work, it is a good idea to load it with the tight package option to reduce
% the amount of white space around the subfigures. subfigure.sty is already
% installed on most LaTeX systems. The latest version and documentation can
% be obtained at:
% http://www.ctan.org/tex-archive/obsolete/macros/latex/contrib/subfigure/
% subfigure.sty has been superceeded by subfig.sty.



%\usepackage[caption=false]{caption}
%\usepackage[font=footnotesize]{subfig}
% subfig.sty, also written by Steven Douglas Cochran, is the modern
% replacement for subfigure.sty. However, subfig.sty requires and
% automatically loads Axel Sommerfeldt's caption.sty which will override
% IEEEtran.cls handling of captions and this will result in nonIEEE style
% figure/table captions. To prevent this problem, be sure and preload
% caption.sty with its "caption=false" package option. This is will preserve
% IEEEtran.cls handing of captions. Version 1.3 (2005/06/28) and later 
% (recommended due to many improvements over 1.2) of subfig.sty supports
% the caption=false option directly:
%\usepackage[caption=false,font=footnotesize]{subfig}
%
% The latest version and documentation can be obtained at:
% http://www.ctan.org/tex-archive/macros/latex/contrib/subfig/
% The latest version and documentation of caption.sty can be obtained at:
% http://www.ctan.org/tex-archive/macros/latex/contrib/caption/




% *** FLOAT PACKAGES ***
%
%\usepackage{fixltx2e}
% fixltx2e, the successor to the earlier fix2col.sty, was written by
% Frank Mittelbach and David Carlisle. This package corrects a few problems
% in the LaTeX2e kernel, the most notable of which is that in current
% LaTeX2e releases, the ordering of single and double column floats is not
% guaranteed to be preserved. Thus, an unpatched LaTeX2e can allow a
% single column figure to be placed prior to an earlier double column
% figure. The latest version and documentation can be found at:
% http://www.ctan.org/tex-archive/macros/latex/base/



%\usepackage{stfloats}
% stfloats.sty was written by Sigitas Tolusis. This package gives LaTeX2e
% the ability to do double column floats at the bottom of the page as well
% as the top. (e.g., "\begin{figure*}[!b]" is not normally possible in
% LaTeX2e). It also provides a command:
%\fnbelowfloat
% to enable the placement of footnotes below bottom floats (the standard
% LaTeX2e kernel puts them above bottom floats). This is an invasive package
% which rewrites many portions of the LaTeX2e float routines. It may not work
% with other packages that modify the LaTeX2e float routines. The latest
% version and documentation can be obtained at:
% http://www.ctan.org/tex-archive/macros/latex/contrib/sttools/
% Documentation is contained in the stfloats.sty comments as well as in the
% presfull.pdf file. Do not use the stfloats baselinefloat ability as IEEE
% does not allow \baselineskip to stretch. Authors submitting work to the
% IEEE should note that IEEE rarely uses double column equations and
% that authors should try to avoid such use. Do not be tempted to use the
% cuted.sty or midfloat.sty packages (also by Sigitas Tolusis) as IEEE does
% not format its papers in such ways.





% *** PDF, URL AND HYPERLINK PACKAGES ***
%
%\usepackage{url}
% url.sty was written by Donald Arseneau. It provides better support for
% handling and breaking URLs. url.sty is already installed on most LaTeX
% systems. The latest version can be obtained at:
% http://www.ctan.org/tex-archive/macros/latex/contrib/misc/
% Read the url.sty source comments for usage information. Basically,
% \url{my_url_here}.





% *** Do not adjust lengths that control margins, column widths, etc. ***
% *** Do not use packages that alter fonts (such as pslatex).         ***
% There should be no need to do such things with IEEEtran.cls V1.6 and later.
% (Unless specifically asked to do so by the journal or conference you plan
% to submit to, of course. )


% correct bad hyphenation here
\hyphenation{op-tical net-works semi-conduc-tor}


\begin{document}
%
% paper title
% can use linebreaks \\ within to get better formatting as desired
\title{A scalable sparse matrix-vector multiplication architecture with Accumulo and D4M}


% author names and affiliations
% use a multiple column layout for up to three different
% affiliations
\author{\IEEEauthorblockN{Yin Huang}
\IEEEauthorblockA{Computer Science and\\Electrical Engineering\\
University of Maryland, Baltimore County\\
Baltimore, MD, 21220\\
Email: yhuang9@umbc.edu}
\and
\IEEEauthorblockN{Yelena Yesha}
\IEEEauthorblockA{Computer Science and\\Electrical Engineering\\
UMBC\\
Baltimore, MD, 21220\\}
\and
\IEEEauthorblockN{Shujia Zhou}
\IEEEauthorblockA{Computer Science and\\Electrical Engineering\\
UMBC\\
Baltimore, MD, 21220}
}

% conference papers do not typically use \thanks and this command
% is locked out in conference mode. If really needed, such as for
% the acknowledgment of grants, issue a \IEEEoverridecommandlockouts
% after \documentclass

% for over three affiliations, or if they all won't fit within the width
% of the page, use this alternative format:
% 
%\author{\IEEEauthorblockN{Michael Shell\IEEEauthorrefmark{1},
%Homer Simpson\IEEEauthorrefmark{2},
%James Kirk\IEEEauthorrefmark{3}, 
%Montgomery Scott\IEEEauthorrefmark{3} and
%Eldon Tyrell\IEEEauthorrefmark{4}}
%\IEEEauthorblockA{\IEEEauthorrefmark{1}School of Electrical and Computer Engineering\\
%Georgia Institute of Technology,
%Atlanta, Georgia 30332--0250\\ Email: see http://www.michaelshell.org/contact.html}
%\IEEEauthorblockA{\IEEEauthorrefmark{2}Twentieth Century Fox, Springfield, USA\\
%Email: homer@thesimpsons.com}
%\IEEEauthorblockA{\IEEEauthorrefmark{3}Starfleet Academy, San Francisco, California 96678-2391\\
%Telephone: (800) 555--1212, Fax: (888) 555--1212}
%\IEEEauthorblockA{\IEEEauthorrefmark{4}Tyrell Inc., 123 Replicant Street, Los Angeles, California 90210--4321}}




% use for special paper notices
%\IEEEspecialpapernotice{(Invited Paper)}




% make the title area
\maketitle


\begin{abstract}
%\boldmath
The increasing volume and velocity of large unstructured datasets have been calling for new technologies to store, query, and analyze data of interest. On one hand, NoSQL distributed databases have been devised to meet the need of big data. On the other hand, scalable linear algebra operations on sparse large matrix has become more and more important. Because most datasets can be represented as a sparse large graph. \\
This paper presents a novel scalable architecture for analyzing massive graphs, with a focus on computing sparse matrix-vector multiplication (SpMV). Our architecture provides scalable linear algebra operations building on recently-developed technologies such as Accumulo, D4M and pMatlab. We store the data in Dynamic Distributed Dimensional Data Model(D4M) format for easy extraction from Accumulo database while pMatlab serves as a parallel computation engine. The principal analysis algorithm is Lanczos-SO for calculating top $k$ eigenvalues and eigenvectors of a matrix. 
Experiments on Graph500 benchmark datasets demonstrate the scalability and efficiency of our architecture.    

\end{abstract}
% IEEEtran.cls defaults to using nonbold math in the Abstract.
% This preserves the distinction between vectors and scalars. However,
% if the conference you are submitting to favors bold math in the abstract,
% then you can use LaTeX's standard command \boldmath at the very start
% of the abstract to achieve this. Many IEEE journals/conferences frown on
% math in the abstract anyway.

% no keywords




% For peer review papers, you can put extra information on the cover
% page as needed:
% \ifCLASSOPTIONpeerreview
% \begin{center} \bfseries EDICS Category: 3-BBND \end{center}
% \fi
%
% For peerreview papers, this IEEEtran command inserts a page break and
% creates the second title. It will be ignored for other modes.
\IEEEpeerreviewmaketitle

\section{Introduction}
Typical data analytics normally include the following pipeline: collecting data, querying data, analyzing data and report. Nowadays Hadoop plays a fundamental role in tackling the challenges caused by increasing volume and velocity of unstructured data. The success of Hadoop relies on its two components: Hadoop Distributed File System (HDFS) for storing massive amount of data with redundancy for failure tolerance and MapReduce for batch processing. \\

In many applications it is intuitive to represent data as a graph to discover patterns hidden underneath. Graph representation has a wide range of applications from social sciences to physics and bio-informatics . Take social media for example, a sparse adjacent matrix for all Twitter users can be built to reflect their relationships. Construction of such a matrix, however, requires complicated operations. Moreover, to find users of similar interests, we need apply linear algebra operations to this matrix. Recent work has focused on constructing graphs from the data stored in the D4M format and applying eigendecomposition to the modularity matrix. "\cite{A scalable signal processing by Benjamin}". In their paper, the authors store their data on a single database node which will become the bottleneck as the data size increases. Our architecture differs from theirs in that we deploy Accumulo, a distributed NoSQL database, as our data storage. \\

Sparse matrix-vector multiplication (SpMV) is of great importance in sparse linear algebra given the fact that they represent the dominant cost in many iterative methods for solving large-scale linear systems and eigenvalue problems that arise in a wide variety of scientific and engineering applications. "\cite{Eficient Sparse Matrix-Vector Multiplication on CUDA}" For example, SpMV is the most expensive operation in Lanczos-SO algorithm employed in HEIGEN which is an eigensolver for billion-scale graphs based on Hadoop. MapReduce, however, is not the best approach for iterative algorithms due to the intermediate shuffling of data among work nodes. "\cite{HEIGEN}" \\

In this paper, we introduce a scalable massive graph analysis architecture integrating D4M and Accumulo with the focus on SpMV. This architecture encompasses the entire data analytics pipeline, from data collection to data extraction of relational structure to data analysis of the resulting graph. More important, we exploit the statistics information from Accumulo table to balance the load and distribute the load evenly among work nodes by issuing queries to fetch rows of a matrix from database table. In addition, we use pMatlab to do parallel processing. In the end, we present and compare our experimental results on Lanczos-SO algorithm against HEIGEN. Our platform shows almost twice speed-up and great scalability.  \\

The rest of the paper is organized as follows. Section 2 describes the architecture, discusses the data storage format and the graph construction procedure. Section 3 explains Lanczos-SO algorithm. Section 4 focuses on D4M and Accumulo. Section 5 demonstrates our implementation of eigensolver using D4M and Accumulo. Section 6 is for experimental results and discussion and section 7 comes to a conclusion. 


\section{System architecture}
Our system consists of the following 3 components: First, the bottom layer is Accumulo database where the data are stored in the D4M format, which provides an easy to use interface for accessing subsets of data. We can thus build graphs representing various types of relationships. Second is the service layer containing Matlab and MatlabMPI, both of which provide the computation resource to the upper layer. Third is the user layer where associative arrays query and store the data to be processed while pMatlab handles the parallel computation. 

\section{Relevant work}
To be completed

\section{Conclusion}
To be completed


% conference papers do not normally have an appendix


% use section* for acknowledgement
\section*{Acknowledgment}


The authors would like to thank...





% trigger a \newpage just before the given reference
% number - used to balance the columns on the last page
% adjust value as needed - may need to be readjusted if
% the document is modified later
%\IEEEtriggeratref{8}
% The "triggered" command can be changed if desired:
%\IEEEtriggercmd{\enlargethispage{-5in}}

% references section

% can use a bibliography generated by BibTeX as a .bbl file
% BibTeX documentation can be easily obtained at:
% http://www.ctan.org/tex-archive/biblio/bibtex/contrib/doc/
% The IEEEtran BibTeX style support page is at:
% http://www.michaelshell.org/tex/ieeetran/bibtex/
%\bibliographystyle{IEEEtran}
% argument is your BibTeX string definitions and bibliography database(s)
%\bibliography{IEEEabrv,../bib/paper}
%
% <OR> manually copy in the resultant .bbl file
% set second argument of \begin to the number of references
% (used to reserve space for the reference number labels box)
\begin{thebibliography}{1}

\bibitem{1}
L.N. Trefethen and D. Bau III, Numberical Linear Algebra, SIAM, 1997.
\bibitem{2} 
U Kang, Breandan Meeder, Evangelos E. Papalexakis, and Christos Faloutsos, HEigen: Spectral Analysis for Billion-Scale graphs, IEEE Transactions on knowledge and data engineering, VOL. 26, No.2, Feb 2014.
\bibitem{3} 
Ankur Dave, Wei Lu, Jared Jackson, Roger Barga, Cloudclustering: Toward an iterative data processing pattern on the cloud.
\bibitem{4} 
Jermey Kepner, William Arcand, etc. DYNAMIC DISTRIBUTED DIMENSIONAL DATA MODEL (D4M) DATABASE AND COMPUTATION SYSTEM
\bibitem{5} 
J.Kepner, Parallel Matlab for Multicore and Multinode computers, SIAM Press, Philadelphia, 2009
\bibitem{6} 
N. Bliss and J. Kepner, pMatlab Parallel Matlab Library, International Journal of High Performance Computing Applications: Special Issue on High Level Programming Languages and Modesl, J.Kepner and H. Zima (editors), Winter 2006 (November)
\bibitem{7} 
J. Kepner and S. Ahalt, “MatlabMPI,” Journal of Parallel and Distributed Computing, vol. 64, issue 8, August, 2004
\bibitem{8} 
N. Bliss, R. Bond, H. Kim, A. Reuther, and J. Kepner, “Interactive Grid Computing at Lincoln Laboratory,” Lincoln Laboratory Journal, vol. 16, no. 1, 2006.


\end{thebibliography}




% that's all folks
\end{document}


